% !TeX spellcheck = ru_RU-Russian
% !TeX encoding = UTF-8 

\documentclass[12pt,a4paper]{article}

\usepackage[russian]{babel}
\usepackage[utf8]{inputenc}
\usepackage{setspace} 
\usepackage[a4paper,
			left=30mm,
			right=10mm,
			top=20mm,
			bottom=20mm]{geometry}
\usepackage{amsmath,amssymb,amsthm, bm}
\usepackage{cite}
\usepackage{graphicx} 
\usepackage{subfigure,subcaption}
\usepackage{kprjHSE} 
\usepackage{tikz}
\usepackage{xcolor,tabularray}
\usepackage{wrapfig}
\usepackage{hyperref}
\hypersetup{
	colorlinks=true,
	linkcolor=black,
	citecolor=black,
	filecolor=black,      
	urlcolor=blue,
}

\usetikzlibrary{positioning}

\renewcommand{\labelenumii}{\arabic{enumi}.\arabic{enumii}}

\lstset{
	frame=single,
	basicstyle=\ttfamily,
	breaklines=true,
	tabsize=4
}

\LabWork
\LabWorkNo{1}
\title{Сегментация и Морфологический анализ текста, Статистика}
\setcounter{MaxMatrixCols}{20}

\FirstAuthor{М.Д.~Кирдин}
\FirstConsultant{Е.И.~Большакова}
\discipline{Компьютерная лингвистика и анализ текстов}
\faculty{Факультет Компьютерных Наук}
\chair{Школа Анализа Данных и Искусственного Интеллекта}
\workyear{2025}

\onehalfspacing

\begin{document}
	\maketitle
	
	\tableofcontents
	
	\begin{introduction}
		
		В данной работе была поставлена цель провести исследование качества автоматического извлечения коллокаций с использованием различных мер ассоциации, при помощи программы на языке \textit{Python}.
		
		Для токенизации, лемматизации, теггинга и синтаксического парсинга был использован модуль \textit{\href{https://spacy.io/}{spaCy}}. Для оценки был использован датасет \textit{\href{https://huggingface.co/datasets/Den4ikAI/russian_cleared_wikipedia}{russian\textunderscore{cleared}\textunderscore{wikipedia}}}, являющийся набором статей на различные темы из русскоязычной Википедии.
	\end{introduction}
	
	\section{Ход работы}

        В данной работе было проведено извлечение коллокаций типов <<A N>> и <<N N>>. Благодаря интерфейсу предлагаемому модулем \textit{spaCy}, достаточно было найти существительные, родителями которых являлись прилагательные или существительные.
        
        Было решено использовать следующие меры:
        \begin{itemize}
            \item $Dice = \dfrac{2f(ab)}{f(a)+f(b)}$,
            \item $MI = \log_2\left({\dfrac{f(ab)\cdot N}{f(a)\cdot f(b)}}\right)$,
            \item $MI^3 = \log_2\left({\dfrac{f^3(ab)\cdot N}{f(a)\cdot f(b)}}\right)$,
            \item $\text{T-score} = \dfrac{f(ab) - \dfrac{f(a)f(b)}{N}}{\sqrt{f(ab)}}$,
        \end{itemize}
        где $f(a), f(b)$ -- абсолютные частоты употребления элементов словосочетания, $f(ab)$ -- абсолютная частота употребления словосочетания и $N$ -- общее число лемм/словоформ в тексте, т.к. статистики для лемм и словоформ рассмотрены раздельно.
		
		С результатами работы программы можно ознакомиться в таблицах 1-4.
        \newpage
        
		\begin{center}
			\textbf{Таблица 1.}~Возможные коллокации при использовании меры $Dice$
			\begin{tblr}{width=\linewidth,
					colspec={|X[c]|X[2, c]|X[c]|X[2, c]|X[c]|}} 
				\hline
				Позиция & Пара лемм & Значение меры & Пара словофром & Значение меры\\
				\hline
				\# 1 & стыдились» наги & 1.000000 & стыдились» наги & 1.000000\\
				\hline
				\# 2 & вальковатый шишками & 1.000000 & вальковатыми шишками & 1.000000\\
				\hline
				\# 3 & судорожный припадки & 1.000000 & декоративном садоводстве & 1.000000\\
				\hline
				\# 4 & сопровождаемому дрожью & 1.000000 & непроходима густа & 1.000000\\
				\hline
				\# 5 & абсентная эпилепсия & 1.000000 & новогодней ёлки & 1.000000\\
				\hline
				\# 6 & субальпийский криволесье & 1.000000 & ноевом ковчеге & 1.000000\\
				\hline
				\# 7 & абазгийская архиепископия & 1.000000 & растительным орнаментам & 1.000000\\
				\hline
				\# 8 & герцинской складчатость & 1.000000 & волнистым узорам & 1.000000\\
				\hline
				\# 9 & сбрасывая рассола & 1.000000 & параноидной шизофрении & 1.000000\\
				\hline
				\# 10 & чехословацкую республику.16 & 1.000000 & сомнительных забегаловках & 1.000000\\
				\hline
				\# 11 & сформированным профсоюзами & 1.000000 & недобросовестными производителями & 1.000000\\
				\hline
				\# 12 & назначаемыми рейхсканцлер & 1.000000 & судорожные припадки & 1.000000\\
				\hline
				\# 13 & передвижной медпункт & 1.000000 & сопровождаемому дрожью & 1.000000\\
				\hline
				\# 14 & бессальниковыми мешалками & 1.000000 & абсентная эпилепсия & 1.000000\\
				\hline
				\# 15 & апериодический ритмической & 1.000000 & платиновой горелки & 1.000000\\
				\hline
				\# 16 & никелированный чайник & 1.000000 & субальпийское криволесье & 1.000000\\
				\hline
				\# 17 & еобъемлющую идеологизация & 1.000000 & кавказский тетерев & 1.000000\\
				\hline
			\end{tblr}
			\newpage
            \begin{tblr}{width=\linewidth,
					colspec={|X[c]|X[2, c]|X[c]|X[2, c]|X[c]|}}
				\hline
				Позиция & Пара лемм & Значение меры & Пара словофром & Значение меры\\
				\hline
				\# 18 & гусаревский овцесовхоз & 1.000000 & абазгийская архиепископия & 1.000000\\
				\hline
				\# 19 & низвергнутые титаны & 1.000000 & абхазскими литераторами & 1.000000\\
				\hline
				\# 20 & ежовый рукавица & 1.000000 & покупательские авансы & 1.000000\\
				\hline
            \end{tblr}
            \vspace{12pt}
            
			\textbf{Таблица 2.}~Возможные коллокации при использовании меры $MI$
			\begin{tblr}{width=\linewidth,
					colspec={|X[c]|X[2, c]|X[c]|X[2, c]|X[c]|}} 
				\hline
				Позиция & Пара лемм & Значение меры & Пара словофром & Значение меры\\
				\hline
				\# 1 & стыдились» наги & 18.992684 & стыдились» наги & 18.992684\\
				\hline
				\# 2 & вальковатый шишками & 18.992684 & вальковатый шишками & 18.992684\\
				\hline
				\# 3 & сопровождаемому дрожью & 18.992684 & непроходима густа & 18.992684\\
				\hline
				\# 4 & абсентная эпилепсия & 18.992684 & новогодней ёлки & 18.992684\\
				\hline
				\# 5 & субальпийский криволесье & 18.992684 & растительным орнаментам & 18.992684\\
				\hline
				\# 6 & абазгийская архиепископия & 18.992684 & волнистым узорам & 18.992684\\
				\hline
				\# 7 & герцинской складчатость & 18.992684 & параноидной шизофрении & 18.992684\\
				\hline
				\# 8 & сбрасывая рассола & 18.992684 & сомнительных забегаловках & 18.992684\\
				\hline
				\# 9 & чехословацкую республику.16 & 18.992684 & недобросовестными производителями: & 18.992684\\
				\hline
				\# 10 & сформированным профсоюзами & 18.992684 & сопровождаемому дрожью & 18.992684\\
				\hline
				\# 11 & назначаемыми рейхсканцлер & 18.992684 & абсентная эпилепсия & 18.992684\\
				\hline
				\# 12 & передвижной медпункт & 18.992684 & платиновой горелки & 18.992684\\
				\hline
				\# 13 & бессальниковыми мешалками & 18.992684 & субальпийское криволесье & 18.992684\\
				\hline
			\end{tblr}
            \newpage
            \begin{tblr}{width=\linewidth,
					colspec={|X[c]|X[2, c]|X[c]|X[2, c]|X[c]|}}
                \hline
				Позиция & Пара лемм & Значение меры & Пара словофром & Значение меры\\
				\hline
				\# 14 & апериодический ритмической & 18.992684 & кавказский тетерев & 18.992684\\
				\hline
				\# 15 & никелированный чайник & 18.992684 & абазгийская архиепископия & 18.992684\\
				\hline
				\# 16 & еобъемлющую идеологизация & 18.992684 & абхазскими литераторами & 18.992684\\
				\hline
				\# 17 & гусаревский овцесовхоз & 18.992684 & покупательские авансы & 18.992684\\
				\hline
				\# 18 & низвергнутые титаны & 18.992684 & стартовые ускорители & 18.992684\\
				\hline
				\# 19 & ежовый рукавица & 18.992684 & подводными лодками & 18.992684\\
				\hline
				\# 20 & неприличие оппортунизм & 18.992684 & летательными аппаратами & 18.992684\\
				\hline
            \end{tblr}
            \vspace{12pt}

            \textbf{Таблица 3.}~Возможные коллокации при использовании меры $MI^3$
			\begin{tblr}{width=\linewidth,
					colspec={|X[c]|X[2, c]|X[c]|X[2, c]|X[c]|}} 
				\hline
				Позиция & Пара лемм & Значение меры & Пара словофром & Значение меры\\
				\hline
				\# 1 & населённый пункт & 24.725356 & xix века & 23.647912\\
				\hline
				\# 2 & xix век & 24.516431 & населённые пункты & 23.410258\\
				\hline
				\# 3 & учебный заведение & 23.840716 & xx века & 23.311953\\
				\hline
				\# 4 & xx век & 23.799354 & мировой войны & 23.091003\\
				\hline
				\# 5 & железный дорога & 23.317423 & известные носители & 22.861682\\
				\hline
				\# 6 & мировой война & 23.258495 & водного реестра & 22.849124\\
				\hline
				\# 7 & xviii век & 22.774249 & второй половине & 22.786880\\
				\hline
				\# 8 & сельский хозяйство & 22.731393 & культурного наследия & 22.667941\\
				\hline
				\# 9 & заработный плата & 22.396550 & сельское хозяйство & 22.657292\\
				\hline
				\# 10 & второй половина & 22.283336 & населённых пунктов & 22.586651\\
				\hline
				\# 11 & xvii век & 22.205239 & железной дороги & 22.467348\\
				\hline				
                \# 12 & вооружённый сила & 21.832522 & стальных труб & 22.458017\\
				\hline
				\# 13 & ванный комната & 21.788099 & российской империи & 22.345181\\
				\hline
				\# 14 & летательный аппарат & 21.657592 & заработная плата & 22.2010544\\
				\hline
				\# 15 & муниципальный образование & 21.589014 & железная дорога & 22.033326\\
				\hline
			\end{tblr}
            \newpage
            \begin{tblr}{width=\linewidth,
					colspec={|X[c]|X[2, c]|X[c]|X[2, c]|X[c]|}} 
				\hline
				Позиция & Пара лемм & Значение меры & Пара словофром & Значение меры\\
				\hline
				\# 16 & xvi век & 21.569048 & географическая характеристика & 22.026254\\
				\hline
				\# 17 & православный церковь & 21.567829 & натуральных чисел & 22.012446\\
				\hline
				\# 18 & полётный палуба & 21.543636 & одномандатным округам & 21.992684\\
				\hline
				\# 19 & слизистый оболочка & 21.508105 & сельском хозяйстве & 21.922896\\
				\hline
				\# 20 & отечественный война & 21.4979875 & бассейновому округу & 21.824084\\
				\hline
			\end{tblr}
            \vspace{12pt}

            \textbf{Таблица 4.}~Возможные коллокации при использовании меры T-score
			\begin{tblr}{width=\linewidth,
					colspec={|X[c]|X[2, c]|X[c]|X[2, c]|X[c]|}} 
				\hline
				Позиция & Пара лемм & Значение меры & Пара словофром & Значение меры\\
				\hline
				\# 1 & xix век & 17.784836 & xix века & 14.777780\\
				\hline
				\# 2 & xx век & 15.399136 & xx века & 13.374665\\
				\hline
				\# 3 & xviii век & 13.023393 & мировой войны & 10.614954\\
				\hline
				\# 4 & мировой война & 12.733342 & xviii века & 9.806472\\
				\hline
				\# 5 & 2010 год & 12.123090 & российской империи & 9.679281\\
				\hline
				\# 6 & больший часть & 11.952826 & второй половине & 9.210340\\
				\hline
				\# 7 & xvii век & 11.825521 & 2011 года & 9.134809\\
				\hline
				\# 8 & 2011 год & 11.421811 & 2010 года & 8.847549\\
				\hline
				\# 9 & второй половина & 10.833149 & второй войны & 8.681058\\
				\hline
				\# 10 & 2009 год & 10.678807 & большая часть & 8.288353\\
				\hline
				\# 11 & xvi век & 10.632479 & главным образом & 8.235930\\
				\hline
				\# 12 & населённый пункт & 10.387935 & населённый пункт & 10.387935\\
				\hline
				\# 13 & российский империя & 10.302256 & российский империя & 10.302256\\
				\hline
				\# 14 & железный дорога & 10.138591 & железный дорога & 10.138591\\
				\hline
				\# 15 & 2014 год & 9.875496 & 2014 год & 9.8754966\\
				\hline
				\# 16 & 2015 год & 9.585918 & 2015 год & 9.585918\\
				\hline
				\# 17 & второй война & 9.547817 & второй война & 9.547817\\
				\hline
				\# 18 & 2008 год & 9.511432 & 2008 год & 9.511432\\
				\hline
				\# 19 & xiii век & 9.499602 & xiii век & 9.499602\\
				\hline
				\# 20 & 2013 год & 9.363738 & 2013 год & 9.363738\\
				\hline
			\end{tblr}
		\end{center}
		
	\begin{results}
		Вследствие того как заданы метрики, по таблицам 1 и 2 можно увидеть, что все коллокации из списка имеют одинаковое значение мер для $Dice$ и $MI$. Это объясняется тем, что пары слов/лемм, употребленные вместе один раз и ни разу не употребленные раздельно получают максимально возможное значение меры. Действительно -- большинство попавших в эти списки коллокаций -- узконаправленные термины, однако стоит заметить что в оба списка также попала устойчивая фраза <<ежовые рукавицы>>. Также в список попали неверно токенизированные пары словоформ: <<чехословацкую республику.16>>, <<стыдились» наги>>.

        Описанная проблема отсутствует в случаях мер $MI^3$ и T-score, т.к. в них количество взаимных взаимных употреблений имеет больший вес и его рост увеличивает значение метрики, а не уменьшает, что легко заметить по таблицам 3 и 4. Стоит заметить, однако, что обе меры часто воспринимают даты или временные промежутки как коллокации: <<xix век>>, <<2015 год>> и т.д.
	\end{results}

	\section*{Заключение}
	\addcontentsline{toc}{section}{\MakeUppercase{Заключение}}
	
	Метрики $Dice$ и $MI$ в силу специфики данных и ошибок токенизации проявили себя хуже всего в выделении коллокация как по леммам, так и по словоформам. Метрики T-score и $MI^3$ проявили себя лучше, при этом выделение колллокаций по леммам оказалось менее верным нежели чем по словоформам, что можно заметить по большей пропорции дат и временных промежутков в таблицах 3 и 4 для лемм. Наконец, мера $MI^3$ справляется с задачей извлечения коллокаций лучше остальных, особенно при подсчете статистик на словоформах. 

	\section*{Приложение А}
	\addcontentsline{toc}{section}{\MakeUppercase{Приложение А}}
	
	\centering\textbf{Листинг А.1.} Программа run.py
	\lstinputlisting[language=Python, showstringspaces=false]{code/run.py}

\end{document}