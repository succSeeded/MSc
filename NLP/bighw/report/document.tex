\documentclass[12pt, a4paper]{article}

\usepackage[russian]{babel}
\usepackage[utf8]{inputenc}
\usepackage{cmap}
\usepackage{setspace} 
\usepackage[a4paper,
			left=30mm,
			right=10mm,
			top=20mm,
			bottom=20mm]{geometry}
\usepackage{amsmath,amssymb,amsthm,bm}
\usepackage{cite}
\usepackage{subfigure,subcaption}
\usepackage{graphicx}
\usepackage{kprjHSE}
\usepackage{tabularray}
\usepackage{hyperref}
\hypersetup{
	colorlinks=true,
	linkcolor=black,
	citecolor=black,
	filecolor=black,      
	urlcolor=blue,
}

\renewcommand{\labelenumii}{\arabic{enumi}.\arabic{enumii}}

\lstset{
	frame=single,
	basicstyle=\ttfamily,
	breaklines=true,
	tabsize=4
}

\LabWork
\title{Реферирование текста на русском языке}
\setcounter{MaxMatrixCols}{20}

\FirstAuthor{М.Д.~Кирдин}
\FirstConsultant{Е.И.~Большакова}
\discipline{Компьютерная лингвистика и анализ текстов}
\faculty{Факультет Компьютерных Наук}
\chair{Школа Анализа Данных и Искусственного Интеллекта}
\workyear{2025}

\onehalfspacing

\begin{document}
	\maketitle

	\tableofcontents
	
	\begin{introduction}
        Автоматическое реферирование текста является одной из основопологающих задач обработки ествественного языка наряду с машинным переводом и распознаванием сущностей. Способы решения этой задачи делятся на две категории: извлекающие и генерирующие. Целью данной работы было провести сравнение этих подходов к решению задачи аннотирования текста. Извлекающие подходы были представлены алгоритмом \textit{TextRank}, а генерирующие подходы --- моделями с трасформерной архитектурой \textit{FRED-T5-Summarize}, а также \textit{rut5-base} с параметрами отрегулированными для решения задачи реферирования текстов на русском языке.  
	\end{introduction}
	
	\section{Ход работы}

    Для сравнения двузх рахличных подходов было решено использовать специализированный датасет, предолженный Ахметгареевой А и др. \cite{dataset}. Он состоит из 197 тыс. текстов в части предназначенной для обучения и 258 текстов поверенных вручную в части для тестов. 

    \subsection{Реализация извлекающего алгоритма}
    Алгоритм \textit{TextRank} является модификацией алгоритма \textit{PageRank}, предложенного \textit{Google} в 1998 году. В данной работе используется вариант данного алогоритма для извлечения предложений. Он основан на построении графа при помощи алгоритма \textit{PageRank}, в котором вершинами являются предложения в тексте и извлечении \textit{n} вершин с наибольшим значением внутренней метрики. В рамках данной работы был написан скрипт на языке \textit{Python} с его реализацией. 

    Для построения графа необходима матрица сходств предложений в реферируемом тексте. Она была получена как набор попарных косинусных расстояний между суммами эмбеддингов отдельных токенов. Эмбеддинги и токенизатор были взяты из библотеки \textit{SpaCy}.

    \subsection{Реализация генерирующих алгоритмов}

    Был написан скрипт на языке \textit{Python}, который  
    \begin{results}
       
	\end{results}

	\section*{Заключение}
	\addcontentsline{toc}{section}{\MakeUppercase{Заключение}}

    \begin{thebibliography}{5}
        \bibitem{dataset}
        Akhmetgareeva A., Kuleshov I., Leschuk V., Abramov A., Fenogenova A., Towards Russian Summarization: can architecture solve data limitations problems? // \url{https://sberlabs.com/publications?publication=1600} (2024).

    \end{thebibliography}
	    
\end{document}
