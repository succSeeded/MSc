\documentclass[12pt, a4paper]{article}

\usepackage[russian]{babel}
\usepackage[utf8]{inputenc}
\usepackage{cmap}
\usepackage{setspace} 
\usepackage[a4paper,
			left=30mm,
			right=10mm,
			top=20mm,
			bottom=20mm]{geometry}
\usepackage{amsmath,amssymb,amsthm,bm}
\usepackage{cite}
\usepackage{subfigure,subcaption}
\usepackage{graphicx}
\usepackage{kprjHSE}
\usepackage{tabularray}
\usepackage{hyperref}
\hypersetup{
	colorlinks=true,
	linkcolor=black,
	citecolor=black,
	filecolor=black,      
	urlcolor=blue,
}

\renewcommand{\labelenumii}{\arabic{enumi}.\arabic{enumii}}

\lstset{
	frame=single,
	basicstyle=\ttfamily,
	breaklines=true,
	tabsize=4
}

\LabWork
\title{Реферирование текста на русском языке}
\setcounter{MaxMatrixCols}{20}

\FirstAuthor{М.Д.~Кирдин}
\FirstConsultant{Е.И.~Большакова}
\discipline{Компьютерная лингвистика и анализ текстов}
\faculty{Факультет Компьютерных Наук}
\chair{Школа Анализа Данных и Искусственного Интеллекта}
\workyear{2025}

\onehalfspacing

\begin{document}
	\maketitle

	\tableofcontents
	
	\begin{introduction}
        Автоматическое реферирование текста является одной из основополагающих задач обработки естественного языка наряду с машинным переводом и распознаванием сущностей. Способы решения этой задачи делятся на две категории: извлекающие и генерирующие. Целью данной работы было провести сравнение этих подходов к решению задачи аннотирования текста. Извлекающие подходы были представлены алгоритмом \textit{TextRank}, а генерирующие подходы --- моделями с трасформерной архитектурой \textit{FRED-T5-Summarize}, а также \textit{rut5-base} с параметрами отрегулированными для решения задачи реферирования текстов на русском языке.  
	\end{introduction}
	
	\section{Ход работы}

    Для сравнения двух различных подходов было решено использовать специализированный датасет, предложенный Ахметгареевой А. и др. \cite{dataset}. Он состоит из 197 тыс. текстов в части предназначенной для обучения и 258 текстов проверенных вручную в части для тестов. Были взяты 40 первых примеров из тестового набора, аннотации, полученные в результате работы сравниваемых алгоритмов, были сохранены в виде текстовых файлов.

    \subsection{Реализация извлекающего алгоритма}
    Алгоритм \textit{TextRank} является модификацией алгоритма \textit{PageRank}, предложенного \textit{Google} в 1998 году. В данной работе используется вариант данного алгоритма для извлечения предложений. Он основан на построении графа при помощи алгоритма \textit{PageRank}, в котором вершинами являются предложения в тексте и извлечении \textit{n} вершин с наибольшим значением внутренней метрики. В рамках данной работы был написан скрипт на языке \textit{Python} с его реализацией, извлекающий ровно \textbf{3 лучших предложения}. 

    Для построения графа необходима матрица сходств предложений в реферируемом тексте. Она была получена как набор попарных косинусных расстояний между суммами эмбеддингов отдельных токенов. Эмбеддинги и токенизатор были взяты из библиотеки \textit{SpaCy}.

    \subsection{Реализация генерирующих алгоритмов}

    Был написан скрипт на языке \textit{Python}, который с помощью моделей \textit{FRED-T5-Summarize} и \textit{rut5-base} с отрегулированными параметрами генерирует реферат текста. Было рассмотрено несколько вариантов генерации ответов:
    
    \begin{itemize}
    	\item с префиксами на английском и декодированием с помощью лучевого поиска;
    	\item с префиксами на русском и декодированием с помощью лучевого поиска;
    	\item с префиксами на английском и декодированием с помощью \textit{promt lookup};
    	\item с префиксами на русском и декодированием с помощью \textit{promt lookup}.
    \end{itemize}
    Со значениям параметров декодирующего слоя можно ознакомиться в \hyperlink{params}{таблице A.1}. Такой выбор вариантов обусловлен тем, что некоторые модели (например, \textit{rut5-base}) являются получены обрезанием многоязычных версий моделей, обученных преимущественно на данных на английском языке, вследствие чего модели могут генерировать лучшие результаты при использовании английских токенов.

    \begin{results}
     	Выбранный извлекающий алгоритм не может производить новые данные, а также использует предложения полностью вследствие чего в некоторых случаях длинные предложения (например, с объемными перечислениями) извлекаются из текста неизменными, что отрицательно влияет на качество аннотации. Например:
     	
     	\begin{center}
     		\begin{tblr}{ 
     				width=\linewidth,
     				colspec={|X[c]|X[c]|} 
     			} 
     			\hline
     			Исходный текст & Аннотация\\
     			\hline
     			Мировой опыт свидетельствует, что для динамичного развития туризма необходимы следующие условия:
     			
     			– стабильная социально-экономическая ситуация (в мире в целом, в отдельной стране и конкретном регионе);
     			
     			– отсутствие административно-чиновничьих барьеров при перемещениях через границы и в период гостевого пребывания;
     			
     			– притягательные рекреационные ресурсы (природно-климатические и культурно-исторические);
     			
     			– развитая инфраструктура туризма и квалифицированные кадры;
     			
     			– высокий уровень сервиса, обеспечение комфортного проживания, гостеприимство, культура и профессионализм персонала;
     			
     			– комфортабельный и безопасный транспорт, надежная связь;
     			
     			– свобода перемещения и гарантии прав путешествующих, обеспечение их безопасности;
     			
     			– высокая ответственность туристских организаций и их структурных подразделений за проведение конкретных туров;
     			
     			– положительный туристский имидж территории, высокая репутация обслуживающих туристов фирм и компаний. & Мировой опыт свидетельствует, что для динамичного развития туризма необходимы следующие условия: 
     			
     			– стабильная социально-экономическая ситуация (в мире в целом, в отдельной стране и конкретном регионе); 
     			
     			– отсутствие административно-чиновничьих барьеров при перемещениях через границы и в период гостевого пребывания; 
     			
     			– притягательные рекреационные ресурсы (природно-климатические и культурно-исторические); 
     			
     			– развитая инфраструктура туризма и квалифицированные кадры; 
     			
     			– высокий уровень сервиса, обеспечение комфортного проживания, гостеприимство, культура и профессионализм персонала; 
     			
     			– комфортабельный и безопасный транспорт, надежная связь; 
     			
     			– свобода перемещения и гарантии прав путешествующих, обеспечение их безопасности; 
     			
     			– высокая ответственность туристских организаций и их структурных подразделений за проведение конкретных туров; 
     			
     			– положительный туристский имидж территории, высокая репутация обслуживающих туристов фирм и компаний.\\
     			\hline
     		\end{tblr}
     	\end{center}
     	
     	Однако в случае если текст достаточно объемный либо состоит из простых предложений качество реферирования значительно лучше:
     	  
     	     	\begin{center}
     		\begin{tblr}{ 
     				width=\linewidth,
     				colspec={|X[c]|X[c]|} 
     			} 
     			\hline
     			Исходный текст & Аннотация\\
     			\hline
     			Тануки - традиционные японские звери-оборотни, символизирующие счастье и благополучие, обычно выглядящие как енотовидные собаки. Второй по популярности зверь-оборотень после кицунэ. В отличие от кицунэ, образ тануки практически лишен негативной окраски. Считается, что тануки — большие любители саке. Поэтому без его присутствия нельзя сделать хорошего сакэ. По этой же причине фигурки тануки, порой весьма большие, являются украшением многих питейных заведений. Они изображают тануки толстяком-добряком с заметным брюшком. & Тануки - традиционные японские звери-оборотни, символизирующие счастье и благополучие, обычно выглядящие как енотовидные собаки. В отличие от кицунэ, образ тануки практически лишен негативной окраски. По этой же причине фигурки тануки, порой весьма большие, являются украшением многих питейных заведений.\\
     			\hline
     		\end{tblr}
     	\end{center}
	\end{results}

	\section*{Заключение}
	\addcontentsline{toc}{section}{\MakeUppercase{Заключение}}

    \begin{thebibliography}{5}
        \bibitem{dataset}
        Akhmetgareeva A., Kuleshov I., Leschuk V., Abramov A., Fenogenova A., Towards Russian Summarization: can architecture solve data limitations problems? // \url{https://sberlabs.com/publications?publication=1600} (2024).

    \end{thebibliography}
    
    \section*{Приложение А}
    \addcontentsline{toc}{section}{\MakeUppercase{Приложение А}}

    \begin{center}
    	\hypertarget{thesentence}{\textbf{Таблица A.1}~Значения параметров декодирующего слоя}
    	\begin{tblr}{ 
    		width=.80\linewidth,
    		colspec={|X[2,l]|X[c]|X[c]|} 
    	} 
    		\hline
    		Параметр & Лучевой поиск & \textit{Promt Lookup}\\
    		\hline
    		количество лучей & 4 & -\\
    		\hline
    		\textit{repetition\textunderscore{}penalty} & 10.0 & 1.5\\
    		\hline
    		\textit{length\textunderscore{}penalty} & 2.0 & -\\
    		\hline
    		модель-помощник & - & Модель без дообучения\\
    		\hline
    		\textit{length\textunderscore{}penalty} & 2.0 & -\\
    		\hline
    		температура & - & 0.4\\
    		\hline
    	\end{tblr}
    \end{center}
	    
\end{document}
